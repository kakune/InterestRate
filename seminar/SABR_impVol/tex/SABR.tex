\documentclass[dvipdfmx,9pt]{beamer}
\usepackage{bxdpx-beamer}
\usepackage{pxjahyper}


% \renewcommand{\kanjifamilydefault}{\gtdefault}
% \setbeamertemplate{navigation symbols}{}
% \usefonttheme{professionalfonts}
% \usetheme{Madrid}


%デザインの選択(省略可)
\usetheme{Luebeck}
%カラーテーマの選択(省略可)
\usecolortheme{orchid}
%フォントテーマの選択(省略可)
\usefonttheme{professionalfonts}
%フレーム内のテーマの選択(省略可)
\useinnertheme{circles}
%フレーム外側のテーマの選択(省略可)
\useoutertheme{infolines}
%しおりの文字化け解消
\usepackage{atbegshi}
\ifnum 42146=\euc"A4A2
\AtBeginShipoutFirst{\special{pdf:tounicode EUC-UCS2}}
\else
\AtBeginShipoutFirst{\special{pdf:tounicode 90ms-RKSJ-UCS2}}
\fi
%ナビゲーションバー非表示
\setbeamertemplate{navigation symbols}{}
%既定をゴシック体に
\renewcommand{\kanjifamilydefault}{\gtdefault}
%タイトル色
\setbeamercolor{title}{fg=structure, bg=}
%フレームタイトル色
\setbeamercolor{frametitle}{fg=structure, bg=}
%スライド番号のみ表示
%\setbeamertemplate{footline}[frame number]
%itemize
\setbeamertemplate{itemize item}{\small\raise0.5pt\hbox{$\bullet$}}
\setbeamertemplate{itemize subitem}{\tiny\raise1.5pt\hbox{$\blacktriangleright$}}
\setbeamertemplate{itemize subsubitem}{\tiny\raise1.5pt\hbox{$\bigstar$}}
% color
\newcommand{\red}[1]{\textcolor{red}{#1}}
\newcommand{\green}[1]{\textcolor{green!40!black}{#1}}
\newcommand{\blue}[1]{\textcolor{blue!80!black}{#1}}

\usepackage{docmute}
\usepackage{graphicx}
\usepackage{dcolumn}
\usepackage{bm}
\usepackage{physics}
\usepackage{amsmath}
\usepackage[version=3]{mhchem}
\usepackage{subfig}
\usepackage{color}
\usepackage{ifthen}
\usepackage{mathtools}
\captionsetup[figure]{font=small,labelfont=bf,justification=raggedright,format=plain}


\newcommand{\exps}[1]{\mathrm{e}^{ #1 }}
\newcommand{\expi}[1]{\exp\pqty{ #1 }}
\newcommand{\exv}[2][t]{\mathbb{E}_{ #1 } \left\lbrack #2 \right\rbrack}


\title{SABR ModelのImplied Volatility}

\author{kakune}
\subject{\LaTeX{}+Beamer}

\begin{document}
\begin{frame}
  \titlepage
\end{frame}

\begin{frame}<beamer>
  \frametitle{Agenda}
  \tableofcontents
\end{frame}

\begin{frame}{Introduction}
  \begin{alertblock}{本スライドの目的と注意}
    \begin{itemize}
      \item 本スライドは、Volatilityが変動するモデルにおける、Implied Volatilityの概念・振る舞い・近似式の導出を理解することを目的とする。
      \item 特に近似式の導出は、よい計算練習になる。
      \item もともと手書きノートで行う予定だったが、やたらmessyになってしまったので、補助としてこのスライドを作成した。
      \item すべての確率変数は$T$-Forward measureの下であるとする。
      \item このスライドは、主に N. Zhang "Properties of the SABR model"に基づく。
    \end{itemize}
  \end{alertblock}
\end{frame}

\section{Introduction}
\begin{frame}{Introduction}
  \begin{block}{Implied Volatility}
    \begin{itemize}
      \item Black-Scholesモデルでは、assetのForward価格は、以下のSDEに従う。
            \begin{equation}
              \dd F_{t} = \sigma^{\mathrm{BS}} \dd W_{t}
            \end{equation}
      \item このモデルを仮定すれば、市場のオプション価格から、quoteするvolを逆算できる。
      \item これを\textbf{Implied Volatility (IV)}と呼ぶ。
    \end{itemize}
  \end{block}
  \begin{block}{BSモデルの限界と改良}
    \begin{itemize}
      \item しかし、実際に計算してみると、オプションのストライクに依存してIVが変化することがわかる。
            \begin{itemize}
              \item Volatility Skew や Volatility Smile。
            \end{itemize}
      \item すなわち、Black-Scholesモデルは市場を再現できていないことになる。
      \item そこで、よりよいモデルとして、Volatilityが変動するモデルが考案された。
            \begin{itemize}
              \item \textbf{Local Volatility (LV) Model}: Volが現在価格のdeteministicな関数とするモデル。
              \item \textbf{Stochastic Volatility (SV) Model}: Volが確率変動するモデル。
            \end{itemize}
    \end{itemize}
  \end{block}
\end{frame}

\section{Black-Scholes Model}
\begin{frame}{Black-Scholes Model}
  \begin{block}{Model}
    Forward price $F_{t}(T)$が、
    \begin{equation}
      \dd F_{t}(T) = \sigma^{\mathrm{BS}} \dd W_{t}
    \end{equation}
    に従うとするModel\footnote{以降、$(T)$はしばしば省略する。}。
    これは解くことができて、 ($s \leq t\leq T$)として$F_{t}(T) = F_{s}(T)\expi{\sigma^{\mathrm{BS}}(W_{t} - W_{s})-\frac{\pqty{\sigma^{\mathrm{BS}}}^{2}}{2}(t-s)}$となる。
  \end{block}
  \begin{block}{European Call Optionの価格}
    時刻$T>0$に行使し、時刻$\tau >T$で受け取るstrike $K$のeuropean call optionの時刻$t$での価格は、
    \begin{equation}
      C_{t}(T,\tau) = P_{t}(t,\tau)\exv[t]{\pqty{F_{t}(T) - K}^{+}}
    \end{equation}
    ブラウン運動の確率分布を考えて解くと、
    \begin{align}
      C_{t}(T,\tau) & = P_{t}(t,\tau) \pqty{F_{t} N(d_{t}^{+}) - KN(d_{t}^{-})}                                                                    \\
      d_{t}^{\pm}   & = \frac{1}{\sigma^{\mathrm{BS}}\sqrt{T-t}}\pqty{\log\pqty{\frac{F_{t}}{K}}\pm\frac{\pqty{\sigma^{\mathrm{BS}}}^{2}}{2}(T-t)}
    \end{align}
    この価格は、$\sigma^{\mathrm{BS}}$について単調増加である。
  \end{block}
\end{frame}

\section{Local Volatility Model}
\begin{frame}{Local Volatility Model}
  \begin{block}{Model}
    Forward price $F_{t}(T)$が、
    \begin{equation}
      \dd F_{t}(T) = \sigma^{\mathrm{loc}}(t,F_{t}(T)) F_{t} \dd W_{t}
    \end{equation}
    に従うとするModel\footnote{C++のコードでは、汎用性のため、係数の$F_{t}$を入れていない。}。ここで、$\sigma^{\mathrm{loc}}$はdeterministicな関数。
  \end{block}
  \begin{block}{European Call Optionの価格}
    当然一般に解くことはできないので、$\sigma^{\mathrm{loc}} = \alpha(t)A(f)/f$、$Q(t,f) = \exv[]{(F_{T} - K)^{+}|F_{t}=f}$として、以下のような手順で解く。
    \begin{enumerate}
      \item Feynman-Kacにより、$Q$の満たす偏微分方程式を求める。
      \item 変数変換を行い、$\tilde{Q}(\tau,x)$とする。
      \item $\tilde{Q},A$に対して漸近展開を行う。
      \item 漸近展開を行った各次数に対して偏微分方程式を導く。
    \end{enumerate}
  \end{block}
\end{frame}

\begin{frame}
  \begin{block}{IVの求め方}
    さらに、BSモデルとの比較を行うことにより、implied volatilityを求める。
    \begin{enumerate}
      \setcounter{enumi}{4}
      \item 漸近形を整理して、(実質的に)一変数関数にする。
      \item BSモデルにおける、その変数の値を計算する。
      \item LVモデルでの変数を計算し、それをBSモデルのものと比較することで、IVを求める。
    \end{enumerate}
    具体的に計算すると、
    \begin{equation}
      \sigma^{\mathrm{IV}} \simeq \sigma^{\mathrm{loc}}\pqty{\frac{f+K}{2}}\pqty{1 + \frac{\sigma^{\mathrm{\mathrm{loc}}~\prime\prime}\pqty{\frac{f+K}{2}}}{\sigma^{\mathrm{loc}}\pqty{\frac{f+K}{2}}}}
    \end{equation}
  \end{block}
\end{frame}

\begin{frame}{LVの限界}
  \begin{block}{LVの限界}
    Marketで観測されるIVが$\sigma^{\mathrm{M}}(F_{0},K)$であるとする。現在のMarketに合うように最低次を考えて$\sigma^{\mathrm{loc}}$のcalibrationを行うと、
    \begin{equation}
      \sigma^{\mathrm{M}}(F_{0},K) = \sigma^{\mathrm{loc}}\pqty{\frac{F_{0}+K}{2}}
    \end{equation}
    となる。ここで、calibrationの直後にforward priceが$F_{0} \to f$と変化したとしよう。すると、そのときのVolは、
    \begin{equation}
      \sigma^{\mathrm{loc}}\pqty{\frac{f+K}{2}} = \sigma^{\mathrm{M}}(F_{0},K+f-F_{0})
    \end{equation}
    となる。例えば$f>F_{0}$とすれば、Volatility curveは左に動くことになる。これは、期待される動きと逆である(Forward priceが右に動いたのだから、curveも右に動いてほしい)。以上が、LVが将来のVolを再現できないと言われる所以である。
  \end{block}
\end{frame}

\section{SABR Model}
\subsection{Modelの概要}
\begin{frame}{Stochastic Volatility Model}
  \begin{block}{Model}
    Forward price $F_{t}(T)$が、
    \begin{equation}
      \dd F_{t}(T) = \sigma_{t} F_{t} \dd W_{t}
    \end{equation}
    に従うとするModel\footnote{C++のコードでは、汎用性のため、係数の$F_{t}$を入れていない。}。ここで、$\sigma_{t}$はあるSDEに従う確率変数。
  \end{block}
  \begin{block}{SABR Model}
    Forward price $F_{t}(T)$が、
    \begin{align}
      \dd F_{t}(T)                & = \alpha_{t} F_{t}^{\beta} \dd W_{t}^{1} \\
      \dd \alpha_{t}              & = \nu \alpha_{t} \dd W_{t}^{2}           \\
      \dd W_{t}^{1} \dd W_{t}^{2} & = \rho \dd t
    \end{align}
    に従うとするModel。
  \end{block}
\end{frame}

\subsection{SABRのIVの漸近形}
\begin{frame}{SABRのIVの漸近形}
  \begin{block}{European Call Option価格導出の概要}
    Volが小さいとして、一般に、
    \begin{align}
      \dd F_{t}      & = \ep \alpha_{t} C(F_{t}) \dd W_{t}^{1} \\
      \dd \alpha_{t} & = \ep \nu \alpha_{t} \dd W_{t}^{2}
    \end{align}
    に対して、Option価格を以下のように導く。
    \begin{enumerate}
      \item $F_{t},\alpha_{t}$の同時確率分布$p$について、Focker-Planckの議論でPDEを導く。
      \item Option価格を、$p$のあるの汎関数$P$で表す。
      \item この$P$を変数変換と漸近展開を繰り返して解く。
    \end{enumerate}
  \end{block}
  \begin{block}{IVの導出の概要}
    \begin{enumerate}
      \setcounter{enumi}{3}
      \item Normal Model $C(f) = 1, \ep\alpha_{t} = \sigma^{\mathrm{N}},\nu = 0$に対して漸近展開形を求める。
      \item Black-Scholes Model $C(f) = f, \alpha_{t} = \sigma^{\mathrm{BS}},\nu = 0$に対して漸近展開形を求め、前項の内容と比較して対応する$\sigma^{\mathrm{N,BS}}$を求める。
      \item SABR Modelに対しても同様にして、$\sigma^{\mathrm{N,SABR}}$を求める。
      \item $\sigma^{\mathrm{N,SABR}} = \sigma^{\mathrm{N,BS}}$を解いて、IVを求める。
    \end{enumerate}
  \end{block}
\end{frame}

\begin{frame}{SABRのIVの漸近形}
  \begin{block}{変数変換の流れ}
    $\raise0.2ex\hbox{\textcircled{\scriptsize{3}}}$について、繰り返し変数変換が行われるので、まとめておく。
    \begin{enumerate}
      \setbeamertemplate{enumerate item}{(\arabic{enumi})}
      \item
            \begin{equation}
              P(\tau ; f,\alpha,K) := \int_{\infty}^{\infty} A^{2} p(t,f,\alpha ; t + \tau ,K,A) \dd A
            \end{equation}
            は次のPDEを満たす︰
            \begin{equation}
              \pdv{P}{\tau} = \frac{\ep^{2}\alpha^{2}C(f)^{2}}{2}\pdv[2]{P}{f} + \ep^{2} \rho \nu \alpha^{2} C(f) \pdv{P}{f}{\alpha} + \frac{\ep^{2}\nu^{2}\alpha^{2}}{2}\pdv[2]{P}{\alpha}
            \end{equation}
      \item
            \begin{equation}
              z = \frac{1}{\ep\alpha}\int^{f}_{K}\frac{\dd g}{C(g)} , \quad \hat{P}(\tau,z,\alpha) := P(\tau,f,\alpha,K)
            \end{equation}
      \item \begin{equation}
              H(\tau,z,\alpha) := \sqrt{\frac{C(K)}{C(f)}}\hat{P}(\tau,z,\alpha)
            \end{equation}
    \end{enumerate}
  \end{block}
\end{frame}

\begin{frame}{SABRのIVの漸近形}
  \begin{block}{変数変換の流れ(続き)}
    \begin{enumerate}
      \setbeamertemplate{enumerate item}{(\arabic{enumi})}
      \setcounter{enumi}{3}
      \item
            \begin{equation}
              \hat{H}(\tau,z,\alpha) := \expi{-\frac{\ep^{2}\rho\nu\alpha b^{(1)}z^{2}}{4}}H(\tau,z,\alpha)
            \end{equation}
      \item
            \begin{equation}
              x := \frac{1}{\ep\nu}\int_{0}^{\ep\nu z}\frac{\dd \zeta}{I(\zeta)}, \quad
              I(\zeta) := \sqrt{1 - 2\rho\zeta + \zeta^{2}}, \quad
              Q(\tau, x) := \frac{\hat{H}(\tau,z,\alpha)}{\sqrt{I(\ep\nu z(x))}}
            \end{equation}
    \end{enumerate}
  \end{block}
\end{frame}

\end{document}
